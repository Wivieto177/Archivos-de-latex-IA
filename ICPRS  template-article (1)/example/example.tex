\documentclass[10pt]{article}

\input{../template/preamble.tex}

%other package

% vectorial font
\usepackage{lmodern}
\usepackage{hyperref}

\usepackage{graphicx}
\usepackage{times}

\begin{document}
\noindent

% This should produce references in the order they appear
\bibliographystyle{ieeetr}

\title{La Inteligencia Artificial en los Videojuegos: Comparativa entre GTA San Andreas y Red Dead Redemption 2}

\authorname{Resumido por Willian Javier Cedeño Bravo*}

\authoraddr{*Universidad Autónoma de Sinaloa, Facultad de Informática Mazatlán, MÉXICO.}

\maketitle

\noindent
\href{https://github.com/Wivieto177/Archivos-de-latex-IA.git}{Repositorio GitHub: https://github.com/Wivieto177/Archivos-de-latex-IA.git}



\section{Introducción}
La inteligencia artificial (IA) ha sido uno de los componentes fundamentales para el avance en el realismo y la complejidad en los videojuegos. Desde sus primeras implementaciones en títulos clásicos hasta las simulaciones detalladas de mundos abiertos que vemos en la actualidad, la IA ha desempeñado un papel crucial en la creación de experiencias de juego más inmersivas. En este contexto, dos juegos se presentan como hitos en la evolución de la IA: *GTA San Andreas* y *Red Dead Redemption 2*, ambos de la desarrolladora Rockstar Games.

*GTA San Andreas*, lanzado en 2004, introdujo una IA revolucionaria para su tiempo, permitiendo la simulación de tráfico, peatones y la interacción con un vasto entorno. Por otro lado, *Red Dead Redemption 2* (2018) representa un salto generacional, donde la IA no solo gobierna el comportamiento de los personajes no jugables (PNJ), sino que también integra elementos naturales como la vida silvestre y el clima, lo que contribuye a un ecosistema virtual vibrante y autónomo.

Este artículo tiene como objetivo analizar la evolución de la IA en estos dos videojuegos, centrándose en las diferencias clave entre las capacidades de IA en términos de jugabilidad, realismo e inmersión del jugador. A lo largo de esta investigación, se examinarán diferentes componentes de la IA, tales como el sistema de tráfico, el comportamiento de los PNJ, las interacciones sociales, la reacción ante eventos aleatorios y la complejidad del combate.

\section{Metodología}
Para la comparativa, tres testers jugaron ambos títulos de manera separada y realizaron un análisis exhaustivo de las características de la IA en cada uno de ellos. Se examinaron diferentes áreas, incluyendo el comportamiento de los PNJ en situaciones de combate, la interacción con el jugador, las dinámicas del tráfico y la respuesta de los sistemas de la policía ante las acciones del jugador. Además, se incluyó un análisis de la fauna en *Red Dead Redemption 2*, debido a la importancia que tiene la vida silvestre en la jugabilidad de este título.

Cada tester documentó sus observaciones durante las sesiones de juego, permitiendo así un análisis comparativo de la IA en distintos escenarios. Posteriormente, se realizaron reuniones de discusión para contrastar los resultados y llegar a conclusiones sobre las fortalezas y limitaciones de la IA en cada juego.

\section{Resultados}
\subsection{Características de la IA en *GTA San Andreas*}
En *GTA San Andreas*, la IA fue innovadora para su época, principalmente en la manera en que simulaba una sociedad en un mundo abierto. Los peatones y los vehículos en el juego siguen patrones predefinidos que otorgan vida al entorno urbano y rural. No obstante, estos comportamientos son bastante simples en comparación con los estándares actuales.

Una de las características más notables es el sistema de tráfico. Los vehículos están programados para seguir reglas básicas de tránsito, como detenerse en semáforos o ceder el paso. Sin embargo, en situaciones de congestión o accidentes, los conductores suelen comportarse de manera errática, incapaces de tomar decisiones alternativas, lo que refleja una limitación en la capacidad de planificación de la IA.

El sistema policial es otro componente fundamental del juego. A medida que el jugador comete crímenes, el nivel de búsqueda aumenta, lo que provoca que la policía responda con mayor agresividad. Aunque esta escalada de dificultad añade tensión y desafío, la IA de la policía sigue patrones simples y predecibles, como perseguir al jugador sin utilizar tácticas complejas de cerco o emboscada.

\subsubsection{Interacciones sociales y PNJ en *GTA San Andreas*}
Las interacciones sociales en *GTA San Andreas* son bastante limitadas. Los PNJ pueden responder a las acciones del jugador, como agresiones físicas o verbales, pero las interacciones carecen de profundidad y diversidad. En muchos casos, los PNJ reaccionan de manera predecible, ya sea huyendo o atacando, dependiendo de la provocación. Estas interacciones se basan en un sistema de respuesta inmediata, sin considerar un contexto más amplio o una progresión narrativa más elaborada.

El comportamiento de las pandillas en el juego añade un nivel de complejidad. El jugador puede aliarse con ciertas facciones o enfrentarse a ellas, lo que genera dinámicas de control territorial y enfrentamientos armados. No obstante, estos conflictos siguen siendo en gran medida repetitivos, ya que la IA de los miembros de las pandillas no varía significativamente en términos de estrategia o táctica durante el combate.

\subsubsection{Sistema de tráfico y fallos en la IA}
Un aspecto que limita la inmersión en *GTA San Andreas* es el comportamiento inconsistente de la IA del tráfico. Cuando ocurre un accidente, por ejemplo, los vehículos no son capaces de adaptarse a la situación, lo que genera atascos o colisiones innecesarias. Además, los peatones pueden quedar atrapados en bucles de comportamiento erráticos, como correr hacia una pared sin poder escapar de la situación. Esto refleja las limitaciones de la IA para gestionar eventos aleatorios de manera eficiente.

A pesar de estas limitaciones, *GTA San Andreas* sentó las bases para muchas de las mecánicas de IA que se desarrollarían en juegos futuros, incluyendo sistemas de tráfico, interacción con PNJ y la dinámica de un mundo abierto.

\subsection{Características de la IA en *Red Dead Redemption 2*}
En *Red Dead Redemption 2*, la IA es mucho más avanzada y versátil. Los PNJ tienen comportamientos complejos y muestran una amplia variedad de reacciones contextuales basadas en las acciones del jugador. Este nivel de sofisticación es evidente en las interacciones sociales, donde los personajes pueden participar en diálogos complejos, reaccionar a cambios en el entorno y modificar su comportamiento según el nivel de honor del jugador.

Uno de los aspectos más impresionantes de la IA en *Red Dead Redemption 2* es la vida silvestre. Los animales tienen rutinas de comportamiento que simulan la caza, la búsqueda de agua y la interacción con otros depredadores o presas. Esto añade una capa de realismo al juego, ya que el jugador puede observar cómo los animales interactúan entre sí y con el entorno sin necesidad de su intervención.

\subsubsection{Sistema de honor y decisiones morales}
El sistema de honor en *Red Dead Redemption 2* es una de las mecánicas más innovadoras del juego. Dependiendo de las decisiones morales del jugador, la IA ajusta el comportamiento de los PNJ hacia el personaje principal, Arthur Morgan. Por ejemplo, si el jugador toma decisiones morales positivas, los habitantes del juego serán más amigables, mientras que las acciones negativas resultarán en una mayor hostilidad y desconfianza. Esto permite una experiencia de juego personalizada y añade una dimensión moral a las acciones del jugador.

\subsubsection{Inteligencia artificial en el combate y vida silvestre}
El combate en *Red Dead Redemption 2* también destaca por su realismo. Los enemigos utilizan tácticas avanzadas, como flanqueos, búsqueda de cobertura y emboscadas. Esta capacidad de adaptación a las situaciones de combate hace que los enfrentamientos sean más desafiantes y realistas en comparación con *GTA San Andreas*, donde los enemigos tienden a atacar de manera directa sin utilizar estrategias elaboradas.

Además, la interacción con la vida silvestre es otro componente crucial en *Red Dead Redemption 2*. Los animales reaccionan a la presencia del jugador de manera realista: algunos huyen, mientras que otros, como los depredadores, pueden atacar si se sienten amenazados. Esta IA avanzada en la fauna no solo contribuye al realismo del juego, sino que también tiene un impacto directo en la jugabilidad, ya que los jugadores deben cazar animales para sobrevivir y utilizar sus recursos.

\section{Comparación de la complejidad de la IA}
La evolución de la IA entre *GTA San Andreas* y *Red Dead Redemption 2* es evidente. Mientras que el primero utilizaba sistemas predefinidos y patrones básicos de comportamiento, el segundo incorpora una IA adaptable, capaz de reaccionar a eventos dinámicos y crear una simulación de vida realista. Los PNJ en *Red Dead Redemption 2* muestran una mayor capacidad para tomar decisiones complejas y responder de manera contextual a las acciones del jugador, lo que resulta en una experiencia de juego mucho más inmersiva.

\section{Conclusión}
La evolución de la IA en los videojuegos ha permitido la creación de mundos más ricos y realistas. Mientras que *GTA San Andreas* fue un pionero en la implementación de IA en un entorno de mundo abierto, *Red Dead Redemption 2* lleva esta tecnología a nuevas alturas, con una IA que no solo gobierna el comportamiento de los personajes, sino que también gestiona la vida silvestre y las interacciones sociales de manera compleja y realista. Estos avances no solo mejoran la jugabilidad, sino que también sugieren un futuro en el que la IA continuará jugando un papel crucial en la creación de experiencias de juego más inmersivas y personalizadas.

\end{document}
